\documentclass{article}
\usepackage{amsmath}
\usepackage{amsthm}
\usepackage{amssymb}
\usepackage[utf8]{inputenc}

\newtheorem{prop}{Proposition}

\title{Towards an Efficient and Economic Deductive System of Matching Logic}
\author{FSL group}

\begin{document}
\maketitle
We aim for a Hilbert style deductive system which has a relatively large number of axioms but only a few inference rules. 

\begin{itemize}
\item (K1) $P \to (Q \to P)$
\item (K2) $(P \to (Q \to R)) \to ((P \to Q) \to (P \to R))$
\item (K3) $(\neg P \to \neg Q) \to (Q \to P)$
\item (K4) $\forall x . (P \to Q) \to (P \to \forall x . Q)$ if $x$ does not occur free in $P$
\item (K5) $\forall x . P \to P$ if $x$ does not occur free in $P$
\item (K6) $\forall x . P(x) \to P(y)$
\item (K7) $P = P$
\item (K8) $P_1 = P_2 \to (Q[P_1/x] \to Q[P_2/x])$
\item (K9) $\exists y . Q = y \to (\forall x . P(x) \to P[Q/x])$ if $Q$ is free for $x$ in $P$
\end{itemize}

Inference rules include
\begin{itemize}
\item (Modus Ponens) From $P$ and $P \to Q$, deduce $Q$.
\item (Universal Generalization) From $P$, deduce $\forall x . P$. 
\end{itemize}

\begin{prop}[Deduction Theorem]
If $\Gamma \cup \{P\} \vdash Q$ and the proof does not use $\forall x$-Generalization where $x$ is free in $P$, then $\Gamma \vdash P \to Q$. 
\end{prop}


\end{document}
