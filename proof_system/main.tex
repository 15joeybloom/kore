\documentclass{article}
\usepackage{amsmath}
\usepackage{amsthm}
\usepackage{amssymb}
\usepackage[utf8]{inputenc}
\usepackage{proof}
\usepackage{txfonts}
\usepackage{mathtools}

\newtheorem{prop}{Proposition}

\DeclarePairedDelimiter\ceil{\lceil}{\rceil}
\DeclarePairedDelimiter\floor{\lfloor}{\rfloor}

\def\imp{\to}
\def\land{\mathbin\&}

\title{Towards an Efficient and Economic Deductive System of Matching Logic}
\author{FSL group}



\begin{document}
\maketitle
We aim for a Hilbert style deductive system which has a relatively large number of axioms but only a few inference rules. 

\section{Grammar and extended grammar}

\begin{align*}
P \Coloneqq\  & x \\
       \mid\  & P_1 \to P_2 \\
       \mid\  & \neg P \\
       \mid\  & \forall x . P \\
       \mid\  & \sigma(P_1,\dots,P_n) \\
       \mid\  & P_1 = P_2 \\
 {\tt (*** }\ & \ {\tt  extended  ***)} \\
       \mid\  & P_1 \vee P_2 \\
       \mid\  & P_1 \wedge P_2 \\
       \mid\  & P_1 \leftrightarrow P_2 \\
       \mid\  & \exists x . P \\
       \mid\  & \top \\
       \mid\  & \bot \\
       \mid\  & \ceil*{P} \\
       \mid\  & \floor*{P} \\
       \mid\  & P_1 \subseteq P_2 
\end{align*}


\section{Hilbert proof system}

\begin{itemize}
\item (K1) $P \to (Q \to P)$
\item (K2) $(P \to (Q \to R)) \to ((P \to Q) \to (P \to R))$
\item (K3) $(\neg P \to \neg Q) \to (Q \to P)$
\item (K4) $\forall x . (P \to Q) \to (P \to \forall x . Q)$ if $x$ does not occur free in $P$
\item (K5) $\forall x . P \to P$ if $x$ does not occur free in $P$
\item (K6) $\forall x . P(x) \to P(y)$
\item (K7) $P = P$
\item (K8) $P_1 = P_2 \to (Q[P_1/x] \to Q[P_2/x])$
\item (K9) $\exists y . Q = y \to (\forall x . P(x) \to P[Q/x])$ if $Q$ is free for $x$ in $P$
\end{itemize}

Inference rules include
\begin{itemize}
\item (Modus Ponens) From $P$ and $P \to Q$, deduce $Q$.
\item (Universal Generalization) From $P$, deduce $\forall x . P$. 
\end{itemize}

\begin{prop}[Deduction Theorem]
If $\Gamma \cup \{P\} \vdash Q$ and the proof does not use $\forall x$-Generalization where $x$ is free in $P$, then $\Gamma \vdash P \to Q$. 
\end{prop}

\section{Inference rules}

\paragraph{Axioms}
$$
\infer{\Gamma \vdash A}{\cdot}
$$
where $A$ is an axiom. 

\paragraph{Modus Ponens}
$$
\infer{\Gamma \vdash P}
      {\Gamma \vdash Q \to P
      &\Gamma \vdash Q}
$$

\paragraph{Universal Generalization}
$$
\infer[(\forall x)]{\Gamma \vdash \forall x . P}
      {\Gamma \vdash P}
$$

\paragraph{Closed-Form Deduction Theorem}
$$
\infer{\Gamma \vdash P \to Q}
      {\Gamma \cup \{P\} \vdash Q}
$$
where $P$ is closed.

\paragraph{Inclusion}
$$
\infer{\Gamma \vdash P}{\cdot}
$$
where $P \in \Gamma$.

\paragraph{Conjunction Introduction}
$$
\infer{\Gamma \vdash P \wedge Q}
      {\Gamma \vdash P
      &\Gamma \vdash Q}
$$
\section*{Examples of proof.sty}

\verb|\infer| draws beautiful proof figures easily:

\noindent (1)
$$
\infer{A}{
	\infer{B}{
		B11\land B12\land B13
		&
		B21\land B22\land B23
	}
	&
	C
}
$$

\noindent (2)
$$
\infer{A1\land A2\land A3\land A4\land A5\land A6}{
	\infer{B}{
		B11\land B12\land B13
		&
		B21\land B22\land B23
	}
	&
	C
}
$$

\noindent (3)
$$
\infer{A1\land A2\land A3\land A4\land A5\land A6}{
	C
	&
	\infer{B}{
		B11\land B12\land B13
		&
		B21\land B22\land B23
	}
}
$$

You can use also some variations:

\noindent (4)
$$
\infer[(1)]{A}{
	\infer*{B}{
		B11\land B12\land B13
		&
		B21\land B22\land B23
	}
	&
	C
}
$$

\noindent (5)
$$
\infer*[(1)]{A1\land A2\land A3\land A4\land A5\land A6}{
	\deduce[{\displaystyle \sum}]{B}{
		B11\land B12\land B13
		&
		B21\land B22\land B23
	}
	&
	\deduce{C}{(2)}
}
$$

\noindent (6)
$$
\infer={A}{A \land B \land C}
$$

Here are more practical examples:

\noindent (7)
$$
\infer[(\land I)]{A \land B}{A & B}
\qquad
\infer[(\land E_l)]A{A\land B}
\qquad
\infer[(\land E_r)]B{A\land B}
$$

$$
\infer[(\imp I)]{A \imp B}{
	\infer*{B}{[A]}
}
\qquad
\infer[(\imp E)]{B}{
	A \imp B
	&
	A
}
$$

Some techniques:
Use \verb|\vcenter| for an equation of proofs.

\noindent (8)
$$
\pi = \vcenter{
	\infer{E}{
		A
		&
		\infer{D}{B & C}
	}
}
$$

Use \verb|\kern| to adjust the form of a proof.

\noindent (9)
$$
\infer{E}{
	A
	&
	\infer{D}{B & C} \kern 0.5cm
}
$$


\end{document}
