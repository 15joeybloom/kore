\documentclass{article}
\usepackage{amsmath}
\usepackage{amsthm}
\usepackage{amssymb}
\usepackage[utf8]{inputenc}
\usepackage{proof}
\usepackage{txfonts}
\usepackage{mathtools}

\newtheorem{thm}{Theorem}
\newtheorem{prop}[thm]{Proposition}
\newtheorem{defn}[thm]{Definition}

\DeclarePairedDelimiter\ceil{\lceil}{\rceil}
\DeclarePairedDelimiter\floor{\lfloor}{\rfloor}

\def\imp{\to}
\def\land{\mathbin\&}

\title{Towards an Efficient and Economic Deductive System of Matching Logic}
\author{FSL group}



\begin{document}
\maketitle
We aim for a Hilbert style deductive system which has a relatively large number of axioms but only a few inference rules. 

\section{Grammar and extended grammar}

The formal language $\mathcal{L}$ we use to write matching logic patterns is defined as follows.
\begin{align*}
P \Coloneqq\  & x \\
       \mid\  & P_1 \to P_2 \\
       \mid\  & \neg P \\
       \mid\  & \forall x . P \\
       \mid\  & \sigma(P_1,\dots,P_n) \\
       \mid\  & P_1 = P_2 \\
 {\tt (*** }\ & \ {\tt  extended  ***)} \\
       \mid\  & P_1 \vee P_2 \\
       \mid\  & P_1 \wedge P_2 \\
       \mid\  & P_1 \leftrightarrow P_2 \\
       \mid\  & \exists x . P \\
       \mid\  & P_1 \neq P_2 \\
       \mid\  & \top \\
       \mid\  & \bot \\
       \mid\  & \ceil*{P} \\
       \mid\  & \floor*{P} \\
       \mid\  & P_1 \subseteq P_2 \\
       \mid\  & x \in P
\end{align*}
with the extended grammar defined as
\begin{align*}
&P_1 \vee P_2 \coloneqq \neg P_2 \to P_1 \\
&P_1 \wedge P_2 \coloneqq \neg (\neg P_1 \vee \neg P_2) \\
&P_1 \leftrightarrow P_2 \coloneqq (P_1 \to P_2) \wedge (P_2 \to P_1) \\
&\exists x . P \coloneqq \neg \forall x . \neg P \\
&P_1 \neq P_2 \coloneqq \neg (P_1 = P_2) \\
&\top \coloneqq x_1 = x_1 \\
&\bot \coloneqq x_1 \neq x_1 \\
&\ceil*{P} \coloneqq P \neq \bot \\
&\floor*{P} \coloneqq P = \top \\
&P_1 \subseteq P_2 \coloneqq \floor*{P_1 \to P_2} \\
&x \in P \coloneqq x \subseteq P
\end{align*}

We will extend the grammar to a many-sorted one in the future.

\section{Hilbert proof system}
Axioms in $\mathcal{L}$ are given by the following nine axiom schemata where $P, Q, R$ are arbitrary patterns and $x, y$ are variables.
\begin{itemize}
\item (K1) $P \to (Q \to P)$
\item (K2) $(P \to (Q \to R)) \to ((P \to Q) \to (P \to R))$
\item (K3) $(\neg P \to \neg Q) \to (Q \to P)$
\item (K4) $\forall x . (P \to Q) \to (P \to \forall x . Q)$ if $x$ does not occur free in $P$
\item (K5) $\forall x . P \to P$ if $x$ does not occur free in $P$
\item (K6) $\forall x . P(x) \to P(y)$
\item (K7) $P = P$
\item (K8) $P_1 = P_2 \to (Q[P_1/x] \to Q[P_2/x])$
\item (K9) $\exists y . Q = y \to (\forall x . P(x) \to P[Q/x])$ if $Q$ is free for $x$ in $P$
\item (K10) $\exists x . x$
\end{itemize}

Inference rules include
\begin{itemize}
\item (Modus Ponens) From $P$ and $P \to Q$, deduce $Q$.
\item (Universal Generalization) From $P$, deduce $\forall x . P$. 
\end{itemize}

\begin{thm}[Soundness of $K_\mathcal{L}$]
Theorems of $K_\mathcal{L}$ are valid.
\end{thm}

\begin{prop}[Deduction Theorem]
If $\Gamma \cup \{P\} \vdash Q$ and the proof does not use \mbox{$\forall x$-Generalization} where $x$ is free in $P$, then $\Gamma \vdash P \to Q$. In particular, when $P$ is closed, $\Gamma \cup \{P\} \vdash Q$ implies $\Gamma \vdash P \to Q$.
\end{prop}

\begin{prop}[Tautology]
For any tautology $\mathcal{A}(p_1,\dots,p_n)$ where $p_1,\dots,p_n$ are propositional variables, $$ \vdash \mathcal{A}(P_1,\dots,P_n).$$
\end{prop}

\begin{prop}[More Theorems in $\mathcal{L}$]
\begin{align*}
&\vdash \exists x . x \\
&\vdash \ceil*{x} \\
&\vdash \exists y . x = y \\
&\vdash P_1 = P_2 \to Q[P_1 / x] = Q[P_2 / x]
\end{align*}
\end{prop}
\begin{proof}
(1)
$$
\infer[^{\tt (DEF)}]{\vdash \exists x . x}
{\infer[^{\tt (MP)}]{\vdash \neg \forall x . \neg x}
	{\infer[^{\tt (TAUT)}]{\vdash \neg (x \to \neg x)}{\cdot}
		&{\infer[^{\tt (MP\&K3)}] 
	{\vdash \neg (x \to \neg x) \to \neg \forall x . \neg x}
	{\infer[^{\tt (DEDUCT)}]
	{\vdash (\forall x . \neg x) \to (x \to \neg x)}
	{\infer[^{\tt (MP)}]{\forall x . \neg x \vdash x \to \neg x}
		{{\infer{\forall x . \neg x \vdash \neg x}
{\forall x . \neg x \vdash \neg x
&\forall x . \neg x \vdash \neg x}}
	    &{\infer[^{\tt (K1)}]{\forall x . \neg x \vdash \neg x \to (x \to \neg x)}{\cdot}}
}}}}}}
$$
(3)
$$
\infer[^{\tt (DEF)}]{\vdash \exists y . x = y}
      {\infer[^{\tt (MP)}]{\vdash \neg \forall y . \neg (x = y)}
      	     {\infer[^{\tt (MP\&TAUT)}]{\vdash \neg \neg (x = x)}
      	     	    {\infer[^{\tt (K7)}]{x = x}{\cdot}}
             &\infer[^{\tt (MP\&K3)}]{\vdash \neg \neg (x = x) \to  \neg \forall y . \neg (x = y)}
                    {\infer[^{\tt {(K6)}}]{\vdash \forall y . \neg (x = y) \to  \neg (x = x)}{\cdot}}}}
$$

\end{proof}

Before we prove the adequacy theorem of $K_\mathcal{L}$, we prove some lemmas.

\begin{prop}
If a pattern is valid, then its closure is valid.
\end{prop}

\begin{prop}
If a pattern's closure is a theorem of $K_\mathcal{L}$, then itself is a theorem of $K_\mathcal{L}$, too.
\end{prop}

\begin{prop}
If $P$ is not a theorem of $K_\mathcal{L}$, then $K_\mathcal{L}$ extended with adding $\neg P$ as an axiom is consistent.
\end{prop}

\begin{prop}
If $S$ is a consistent extended system of $K_\mathcal{L}$, then for any theorem $P$ of $S$, there exists a model $M$ and an $M$-evaluation $\rho$ such that $M,\rho \vDash P$.
\end{prop}

\section{Inference rules}

\paragraph{Axioms}
$$
\infer{\Gamma \vdash A}{\cdot}
$$
where $A$ is an axiom. 

\paragraph{Inclusion}
$$
\infer{\Gamma \vdash P}{\cdot}
$$
where $P \in \Gamma$.

\paragraph{Modus Ponens}
$$
\infer{\Gamma \vdash P}
      {\Gamma \vdash Q \to P
      &\Gamma \vdash Q}
$$

\paragraph{Closed-Form Deduction Theorem}
$$
\infer{\Gamma \vdash P \to Q}
{\Gamma \cup \{P\} \vdash Q}
$$
where $P$ is closed.

\paragraph{Universal Generalization}
$$
\infer[(\forall x)]{\Gamma \vdash \forall x . P}
      {\Gamma \vdash P}
$$

\paragraph{Conjunction Splitting}
$$
\infer{\Gamma \vdash P \wedge Q}
      {\Gamma \vdash P
      &\Gamma \vdash Q}
$$

\end{document}
