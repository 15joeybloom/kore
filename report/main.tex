\documentclass{article}
\usepackage{amsmath}
\usepackage{amsthm}
\usepackage{amssymb}
\usepackage[utf8]{inputenc}
\usepackage{proof}
\usepackage{txfonts}
\usepackage{mathtools}

\usepackage{listings}
\usepackage{xcolor}
\lstdefinelanguage{maude}
{
    keywords={sort, sorts, op, ops, var, vars, eq, ceq, mb , endfm, fmod, is, endm, including},
    morecomment=[l]{---}
}
\lstset{
	language=maude,
	basicstyle=\ttfamily,
	keywordstyle=\color{blue}
}

\newtheorem{thm}{Theorem}
\newtheorem{prop}[thm]{Proposition}
\newtheorem{cor}[thm]{Corollary}
\newtheorem{defn}[thm]{Definition}

\DeclarePairedDelimiter\ceil{\lceil}{\rceil}
\DeclarePairedDelimiter\floor{\lfloor}{\rfloor}

\def\imp{\to}
\def\land{\mathbin\&}



\title{Automated Deduction in Matching Logic}
\author{FSL group}



\begin{document}
\maketitle

Recent success in building very fast automated theorem provers (especially for first-order theories) makes us look forward to an highly efficient automated deductive system for matching logic. This project aims at that.

\section{Grammar}

As in most logic, formulas of matching logic, called \emph{patterns}, are written in a formal language, denoted as $\mathcal{L}$, who has a very similar grammar as first order logic. 

The language $\mathcal{L}$ in general is a many-sorted language. A signature of $\mathcal{L}$ contains not only a finite set $\Sigma$ of symbols, but also a finite nonempyty set $S$ of sorts. Each symbol $\sigma \in \Sigma$ is, of course, sorted, with a fixed nonempty arity. We write $\sigma \in \Sigma_{s_1,\dots,s_n,s}$ when we want to emphasize that $\sigma$ takes $n$ arguments (with suggested sorts) and returns a pattern in sort $s$, but we hope in most cases sorting is clear from context.

The basic grammar for $\mathcal{L}$, as defined below, is almost identical to first-order logic, except that in $\mathcal{L}$ there is no difference between relational (predicate) and functional symbols, and we accept first-order terms as patterns in matching logic. 
\begin{align*}
P \Coloneqq\  & x \\
       \mid\  & P_1 \to P_2 \\
       \mid\  & \neg P \\
       \mid\  & \forall x . P \\
       \mid\  & \sigma(P_1,\dots,P_n).
\end{align*}
For simplicity, we did not mention sorting in the grammar definition, and assume it should be clear to all readers. For example, in $P_1 \to P_2$, both patterns $P_1$ and $P_2$ should have the same sort, and that sort is the sort of $P_1 \to P_2$. The sort of $\forall x . P$ is the sort of $P$, where the sort of variable $x$ does not matter. To see why it is the case, consider the pattern $\exists x . list(x, 1 \cdot 3 \cdot 5)$, which is the set of all memory configurations that has a list $(1,3,5)$ in it.

Propositional connectives are always assumed, including conjunction ($\wedge$), disjunction ($\vee$), and equivalence ($\leftrightarrow$). Existential quantifier ($\exists x$) is defined by universal quantifier ($\forall x$) in the normal way. Bottom ($\bot_s$) and top ($\top_s$) in sort $s$ are given by $x \wedge \neg x$ and $\neg \bot_s$, respectively, where $x$ is a variable in sort $s$. It does not matter which variable we pick.

\subsection{Extension}
The formal language is often extended with definedness symbols. A definedness symbol $\ceil*{\_}_{s_1}^{s_2} \in \Sigma_{s_1,s_2}$ is a \emph{predicate} symbol, i.e., $\ceil*{P}_{s_1}^{s_2}$ is either $\top$ or $\bot$ for any pattern $P$. We often write $\ceil*{P}$ when sorting information can be derived from the context.

Definedness extends the formal language:
\begin{align*}
&\floor*{P} \coloneqq {\neg} \ceil*{\neg P} \\
&P_1 = P_2 \coloneqq \floor*{P_1 \leftrightarrow P_2} \\
&P_1 \neq P_2 \coloneqq \neg (P_1 = P_2) \\
&P_1 \subseteq P_2 \coloneqq \floor*{P_1 \to P_2} \\
&x \in P \coloneqq x \subseteq P
\end{align*}
and provides convenient ways to write \emph{predicate patterns}, patterns that is either $\top$ or~$\bot$. The sort of a predicate patterns is often of no importance. It is its ``truth value'' that is important. For example, $\ceil*{P}$ is ``true'' if $P$ is not empty, and this fact can be used in any context. This leads us to polymorphic sorting in matching logic.

\subsection{Polymorphism}
Predicate symbols are often polymorphic sorting. As a result, many predicate connectives are of polymorphic types, too. Predicate pattern $P_1 = P_2$ only requires $P_1$ and $P_2$ have the same sort, and its result can be of any sort. If $P_1$ and $P_2$ have different sorts, $P_1 = P_2$ is \emph{ill-sorted}.

Polymorphism means in mathematics a mapping $\tau$ from pattern set $\mathcal{P}$ to a sort algebra $\mathcal{S}$ built from $S$. The algebra $\mathcal{S}$ is obtained by adding two special elements, called the top-sort $s_\top$ and bottom-sort $s_\bot$, to the set $S$, and define a partial order relation by which $s_\top$ is the largest element in $\mathcal{S}$, $s_\bot$ is the smallest, and every other elements in $S$ is in between, incomparable to each other. $\mathcal{S}$ is a lattice.

Let $P$ be a pattern. Intuitively, if $\tau(P) = s_\bot$ then $P$ is ill-sorted. Most patterns have their normal sorts in $S$. Predicate patterns have sort $s_\top$, which makes them fit in any context. However, nothing prevents one having a non-predicate pattern of sort $s_\top$. All that matters is the \emph{sorting signature} of a symbol $\sigma \in \Sigma$.

Normal symbols have regular sorting signatures. A symbol $\sigma \in \Sigma_{s_1,\dots,s_n, s}$ expects $n$ arguments of the sorts $s_1, \dots, s_n$ respectively, and results in a pattern of sort $s$. This fact is denoted as $\sigma \colon s_1 \times \dots \times s_n \Rightarrow s$.

The same notation can be used to polymorphic symbols, most of which are predicates. For example, $\mathtt{primeQ} \colon \mathtt{Nat} \Rightarrow s_\top$ checks whether its argument is a prime number or not, and returns either $\top$ (if the argument is a prime number) or $\bot$ (otherwise). A two-arity example is $\mathtt{ge}(x,y)$ that checks whether $x > y$, where $\mathtt{ge} \colon \mathtt{Nat} \times \mathtt{Nat} \Rightarrow s_\top$. Predicates can appear in any context, for instance, $\forall x . \exists y . (\mathtt{ge}(y, x) \wedge \mathtt{primeQ}(y))$.

Symbols can take arguments of the top sort $s_\top$, too. One example is the definedness symbol $\ceil*{\_} \colon s_\top \Rightarrow s_\top$, a predicate over \emph{all sorts} of patterns. It sounds more ``polymorphic'' than $\mathtt{ge}$ and $\mathtt{primeQ}$, in the sense that not only it can be plugged in any context, but also any patterns can be plugged in it. The above discussion leads us to the following definition of polymorphism.

\begin{defn}[Polymorphism]
	Polymorphic sort of patterns is a mapping $\tau \colon \mathcal{P} \to \mathcal{S}$, defined recursively by
\begin{align*}
  &\tau(x) = s \quad \text{if $x$ is a variable of sort $s$} \\
  &\tau(\neg P) = \tau(P) \\
  &\tau(P_1 \to P_2) = \min(\tau(P_1), \tau(P_2)) \\
  &\tau(\exists x . P) = \tau(P).
\end{align*}
Let $\sigma \colon s_1 \times \dots s_n \Rightarrow s$ be a polymorphic symbol. 
\begin{equation*}
\tau(\sigma(P_1,\dots,P_n)) =
\begin{cases}
s \quad &\text{if \ $\min(\tau(P_i), s_i) \neq s_\bot$ \ for all $i$} \\
s_\bot \quad &\text{otherwise}
\end{cases}
\end{equation*}
\end{defn}

\subsubsection{Polymorphism in Maude}
Algebra $\mathcal{S}$ has a neat implementation in Maude.

\begin{lstlisting}
fmod POLYSORT is
  including PATTERN .
  sort Sort .  op topsort : -> S [ctor] .
  --- some normal symbols
  ops Nat Seq Map : -> S [ctor] .
  --- tau-polymorphism
  op tau : Pattern -> [Sort] .
endfm
\end{lstlisting}

Notice how we take use of Maude's ``kind'' type to indicate ill-sorted patterns.


\section{Deductive system}
Axioms in $\mathcal{L}$ are given by the following axiom schemata where $P, Q, R$ are arbitrary patterns and $x, y$ are variables.
\begin{itemize}
\item (K1) $P \to (Q \to P)$
\item (K2) $(P \to (Q \to R)) \to ((P \to Q) \to (P \to R))$
\item (K3) $(\neg P \to \neg Q) \to (Q \to P)$
\item (K4) $\forall x . (P \to Q) \to (P \to \forall x . Q)$ if $x$ does not occur free in $P$
\item (K5) $\exists y . x = y$
\item (K6) $\exists y . Q = y \to (\forall x . P(x) \to P[Q/x])$ if $Q$ is free for $x$ in $P$
\item (K7) $P_1 = P_2 \to (Q[P_1/x] \to Q[P_2/x])$
\item (M1) $x \in y = (x = y)$
\item (M2) $x \in \neg P = \neg (x \in P)$
\item (M3) $x \in P \wedge Q = (x \in P) \wedge (x \in Q)$
\item (M4) $x \in \exists y . P = \exists y . x \in P$ where $x$ is distinct from $y$
\item (M5) $x \in \sigma(\dots,P_i,\dots) = \exists y . y \in P_i \wedge x \in \sigma(\dots,y,\dots)$
\end{itemize}

Inference rules include
\begin{itemize}
\item (Modus Ponens) From $P$ and $P \to Q$, deduce $Q$.
\item (Universal Generalization) From $P$, deduce $\forall x . P$. 
\item (Membership Introduction) From $P$, deduce $\forall x . (x \in P)$, where $x$ does not occur free in $P$.
\item (Membership Elimination) From $\forall x . (x \in P)$, deduce $P$, where $x$ does not occur free in $P$.
\end{itemize}

\begin{thm}[Soundness of $K_\mathcal{L}$]
Theorems of $K_\mathcal{L}$ are valid.
\end{thm}
\begin{proof}
	Trivial.
\end{proof}

We provide some metatheorems of $K_\mathcal{L}$.

\begin{prop}[Tautology] \label{prop:taut}
	For any propositional tautology $\mathcal{A}(p_1,\dots,p_n)$ where $p_1,\dots,p_n$ are propositional variables, $$ \vdash \mathcal{A}(P_1,\dots,P_n).$$
\end{prop}
\begin{proof}
	Omit proof here.
\end{proof}
\paragraph{Remark} Proposition~\ref{prop:taut} makes any metatheorem of propositional logic a metatheorem of $K_\mathcal{L}$.
\begin{prop} [Variable Substitution]
	$\vdash \forall x . P \to P[y / x]$.
\end{prop}
\begin{prop}[Functional Substitution]
	$\vdash \exists y . (Q = y) \to (P[Q / x] \to \exists x . P(x))$.
\end{prop}
\begin{prop}[$\vee$-Introduction] \label{prop:vee-introduction}
	$\vdash P$ implies $\vdash P \vee Q$.
\end{prop}
\begin{proof}
Use Proposition~\ref{prop:taut} and Modus Ponens.
Note that in general, $\vdash P \vee Q$ does not imply $\vdash P$ or $\vdash Q$.
\end{proof}

\begin{prop}[$\wedge$-Introduction and Elimination]
\label{prop:wedge}
$\vdash P$ and $\vdash Q$ iff $\vdash P \wedge Q$.
\end{prop}
\begin{proof}
Use Proposition~\ref{prop:taut} and Modus Ponens.
\end{proof}

\begin{prop}[Equality Introduction]
	$\vdash P = P$.
\end{prop}
\begin{proof}
	Use Membership Introduction and Proposition~\ref{prop:taut}.
\end{proof}

\begin{prop}[Equality Replacement]
\label{prop:ereplacement}
$\vdash P_1 = P_2$ and $\vdash Q[P_1 / x]$ implies $\vdash Q[P_2 / x]$.
\end{prop}
\begin{proof}
	Use Axiom~(K7) and Modus Ponens.
\end{proof}

\begin{prop}[Equality Establishment]
\label{prop:eestablish}
$\vdash P \leftrightarrow Q$ implies $\vdash P = Q$.
\end{prop}
\begin{proof}
	Use Membership Axoims and $\vee$-Introduction.
\end{proof}
\begin{cor} \label{cor:top}
$\vdash P$ implies $\vdash P = \top$ .
\end{cor}

\begin{prop} \label{prop:var1}
$\vdash x \in \ceil*{y}$.
\end{prop}
\begin{proof}
\begin{align*}
&\vdash x \in \ceil*{y} \\
{\text{if}}& \vdash \forall x . (x \in \ceil*{y}) \tag{$K5$, $K6$, and Modus Ponens} \\
{\text{iff}}& \vdash \ceil*{y}.
\end{align*}
\end{proof}

\begin{prop} \label{prop:p->ceilp}
$\vdash P \to \ceil*{P}$.
\end{prop}
\begin{proof}
\begin{align*} 
&\vdash P \to \ceil*{P} \\
{\text{iff}}& \vdash \forall x . (x \in P \to \ceil*{P}) \\
{\text{if}}&  \vdash x \in P \to \ceil*{P} \\
{\text{iff}}& \vdash x \in P \to x \in \ceil*{P} \\
{\text{iff}}& \vdash x \in P \to \exists y . (y \in P \wedge x \in \ceil*{y}) \\
{\text{iff}}& \vdash x \in P \to \neg \forall y . (y \not \in P \vee x \not \in \ceil*{y}) \\
{\text{iff}}& \vdash \forall y . (y \not \in P \vee x \not \in \ceil*{y}) \to x \not \in P \\
{\text{if}}&  \vdash x \not \in P \vee x \not \in \ceil*{x} \to x \not \in P \\
{\text{iff}}& \vdash x \in P \to x \in P \wedge x \in \ceil*{x} \\
{\text{iff}}& \vdash x \in P \to x \in \ceil*{x} \\
{\text{if}}&  \vdash x \in \ceil*{x} \\
\end{align*}
\paragraph{Remark} Similarly we can show $\vdash \floor*{P} \to P$.
\end{proof}

\begin{prop} \label{prop:floor}
$\vdash \forall x . (x \in P) = \floor*{P}$, where $x$ occurs free in $P$.
\end{prop}
\begin{proof}
By Proposition~\ref{prop:eestablish} and~\ref{prop:wedge}, it suffices to show \begin{equation} \label{eq:prop11-1}
\vdash \forall x . (x \in P) \to \floor*{P}
\end{equation}
and
\begin{equation} \label{eq:prop11-2}
\vdash \floor*{P} \to \forall x . (x \in P).
\end{equation}

To show~\eqref{eq:prop11-1},

\begin{align*}
&\vdash \forall x . (x \in P) \to \floor*{P} \\
{\text{iff}}& \vdash \forall x . \ceil*{x \wedge P} \to \neg \ceil*{\neg P} \\
{\text{iff}}& \vdash \ceil*{\neg P} \to \exists x . \neg \ceil*{x \wedge P} \\
{\text{iff}}& \vdash \forall y . (y \in ( \ceil*{\neg P} \to \exists x . \neg \ceil*{x \wedge P})) \\
{\text{if}}& \vdash y \in ( \ceil*{\neg P} \to \exists x . \neg \ceil*{x \wedge P} \\
{\text{iff}}& \vdash \exists z_1 . (z_1 \not \in P \wedge y \in \ceil*{z_1}) \to \\ & \quad \quad \exists x . \neg (\exists z_2 . (z_2 = x \wedge z_2 \in P \wedge y \in \ceil*{z_2})) \\
{\text{iff}}& \vdash \exists z_1 . (z_1 \not \in P \wedge \top) \to \tag{Proposition~\ref{prop:var1}, \ref{prop:ereplacement}, and Corollary~\ref{cor:top}}\\
& \quad \quad \exists x . \neg (\exists z_2 . (z_2 = x \wedge z_2 \in P \wedge \top)) \\
{\text{iff}}& \vdash \exists z_1 . (z_1 \not \in P) \to \exists x . \neg (\exists z_2 . (z_2 = x \wedge z_2 \in P)) \\
{\text{iff}}& \vdash \forall x . (\exists z_2 . (z_2 = x \wedge z_2 \in P)) \to \forall z_1 . (z_1 \in P) \\
{\text{if}}& \vdash \forall z_1 . (\forall x . (\exists z_2 . (z_2 = x \wedge z_2 \in P)) \to (z_1 \in P)) \\
{\text{if}}& \vdash \forall x . (\exists z_2 . (z_2 = x \wedge z_2 \in P)) \to (z_1 \in P).
\end{align*}
Since $\vdash \forall x . (\exists z_2 . (z_2 = x \wedge z_2 \in P)) \to \exists z_2 . (z_2 = z_1 \wedge z_2 \in P) $, it suffices to show
\begin{align*}
&\vdash \exists z_2 . (z_2 = z_1 \wedge z_2 \in P) \to (z_1 \in P) \\
{\text{iff}}& \vdash z_1 \not \in P \to \forall z_2 . (z_2 \neq z_1 \vee z_2 \not \in P) \\
{\text{if}}& \vdash \forall z_2 . (z_1 \not \in P \to z_2 \neq z_1 \vee z_2 \not \in P) \\
{\text{if}}& \vdash z_1 \not \in P \to z_2 \neq z_1 \vee z_2 \not \in P \\
{\text{if}}& \vdash z_2 = z_1 \wedge z_2 \in P \to z_1 \in P.
\end{align*}
And we proved~\eqref{eq:prop11-1}.

Similarly, to show~\eqref{eq:prop11-2}, 
\begin{align*}
&\vdash \floor*{P} \to \forall x . (x \in P)  \\
{\text{iff}}& \vdash \exists x . \neg \ceil*{x \wedge P} \to \ceil*{\neg P} \\
{\text{iff}}& \vdash \forall y . (y \in \exists x . \neg \ceil*{x \wedge P} \to \ceil*{\neg P}) \\
{\text{if}}& \vdash y \in \exists x . \neg \ceil*{x \wedge P} \to \ceil*{\neg P} \\
{\text{iff}}& \vdash \exists x . \neg \exists z_2 . (z_2 = x \wedge z_2 \in P) \to \exists z_1 . (z_1 \not \in P) \\
{\text{iff}}& \vdash \forall z_1 . (z_1 \in P) \to \exists z_2 . (z_2 = x \wedge z_2 \in P)\\
{\text{iff}}& \vdash x \in P \to \exists z_2 . (z_2 = x \wedge z_2 \in P).
\end{align*}
We proved~\eqref{eq:prop11-2}.

\paragraph{Remark} If $x$ occurs free in $P$, the result does not hold. For example, let $P$ be $upto(x)$ where $upto(\cdot)$ is interpreted to $upto(n) = \{0, 1, \dots, n\}$ on $\mathbb{N}$. 
\end{proof}

\paragraph{Remark} From Membership Introduction and Elimination inference rules and Proposition~\ref{prop:floor}, $\vdash P$ iff $\vdash \floor*{P}$.

\begin{prop}[Classification Reasoning] \label{prop:classificationreasoning}
	For any $P$ and $Q$, from $\vdash P \to Q$ and $\vdash \neg P \to Q$ deduce $\vdash Q$.
\end{prop}
\begin{proof}
	From $\vdash \neg P \to Q$ deduce $\vdash \neg Q \to P$. Notice that $\vdash P \to Q$, so we have $\vdash \neg Q \to Q$, i.e., $\vdash \neg \neg Q \vee Q$ which concludes the proof.
\end{proof}
\begin{cor} \label{cor:classificationreasoning}
	For any $P_1$, $P_2$, and $Q$ are patterns with $\vdash P_1 \vee P_2$, from $\vdash P_1 \to Q$ and $\vdash P_2 \to Q$, deduce $\vdash Q$.
\end{cor}

\begin{defn}[Predicate Pattern]
A pattern $P$ is called a predicate pattern or a predicate if $\vdash (P = \top) \vee (P = \bot)$.
\end{defn}
\paragraph{Remark} Predicate patterns are closed under all logic connectives.
\paragraph{Remark} For any $P$, $\ceil*{P}$ is a predicate pattern.

\begin{prop} \label{prop:017}
	$\vdash (\ceil*{P} = \bot) = (P = \bot)$ and $\vdash (\floor*{P} = \top) = (P = \top)$.
\end{prop}
\begin{proof}
	It is easy to prove one derivation from the other, so we only prove the first one. By Proposition~\ref{prop:eestablish}, it suffices to prove
	\begin{equation} \label{eq:prop:017:a}
	\vdash (\ceil*{P} = \bot) \to (P = \bot)
	\end{equation}
	and
	\begin{equation} \label{eq:prop:017:b}
	\vdash (P = \bot) \to (\ceil*{P} = \bot)
	\end{equation}
	
	The proof of~\eqref{eq:prop:017:b} is trivial and we left it as an exercise. We now prove~\eqref{eq:prop:017:a} through the following backward reasoning.
	
	\begin{align}
	&\vdash (\ceil*{P} = \bot) \to (P = \bot)  \nonumber \\
	\text{iff} \quad &\vdash \forall y . (y \in ((\ceil*{P} = \bot) \to (P = \bot))) \nonumber \\
	\text{if} \quad &\vdash y \in ((\ceil*{P} = \bot) \to (P = \bot)) \nonumber \\
	\text{iff} \quad &\vdash (y \in (\ceil*{P} = \bot) \to (y \in (P = \bot)). \label{eq:prop:017:a:1}
	\end{align}
	While for any pattern $Q$, 
	\begin{align*}
	&\vdash y \in (Q = \bot) \\
	\text{iff} \quad &\vdash y \in \neg \ceil*{\neg (Q \leftrightarrow \bot)} \\
	\text{iff} \quad &\vdash y \in \neg \ceil*{Q}\\
	\text{iff} \quad &\vdash \neg \exists z . (z \in Q \wedge y \in \ceil*{z}) \\
	\text{iff} \quad &\vdash \neg \exists z . (z \in Q) 
	\end{align*}
	So we continue to prove~\eqref{eq:prop:017:a:1} by showing
	\begin{align*}
&\vdash (y \in (\ceil*{P} = \bot)) \to (y \in (P = \bot))  \\
\text{iff} \quad &\vdash \neg \exists z . (z \in \ceil*{P}) \to \neg \exists z . (z \in P) \\
\text{iff} \quad &\vdash \exists z . (z \in P) \to  \exists z . (z \in \ceil*{P}) \\
\text{iff} \quad &\vdash \exists z . (z \in P) \to \exists z . (\exists z_1 . (z_1 \in P \wedge z \in \ceil*{z_1}))\\
\text{iff} \quad &\vdash \exists z . (z \in P) \to \exists z . \exists z_1 . (z_1 \in P) \\
\text{iff} \quad &\vdash \exists z_1 . (z_1 \in P) \to \exists z . \exists z_1 . (z_1 \in P).
	\end{align*}
	And we finish the proof by noticing the fact that for any pattern $Q$ and variable $x$, 
	$$ \vdash Q \to \exists x . Q.$$
	
\end{proof}



\begin{prop} \label{prop:predicateabsorb}
For any predicate $P$, $\vdash (P \neq \top) = (P = \bot)$ and $\vdash (P \neq \bot) = (P = \top)$.
\end{prop}
\begin{proof}
We only prove the first derivation, by showing both
\begin{equation}\label{eq:predicateabsort1}
\vdash (P \neq \top) \to (P = \bot)
\end{equation}
and
\begin{equation}\label{eq:predicateabsort2}
\vdash (P = \bot) \to (P \neq \top).
\end{equation}
Proving~\eqref{eq:predicateabsort2} is trivial. We now prove~\eqref{eq:predicateabsort1}, which is also trivial by transforming disjunction to implication.
\end{proof}
\begin{prop} \label{prop:predicatevee}
For any pattern $Q$ and any predicate pattern $P$, $\vdash P \vee Q$ iff $\vdash P \vee \floor*{Q}$.
\end{prop}
\begin{proof}
($\Leftarrow$) is obtained immediately by the remark of Proposition~\ref{prop:p->ceilp}. We now prove ($\Rightarrow$). 

Because $\vdash Q = \top \vee Q \neq \top$, it suffices to show
\begin{equation} \label{eq:predicatevee-01}
\vdash Q = \top \to (P \vee \floor*{Q} = \top)
\end{equation}
and
\begin{equation} \label{eq:predicatevee-02}
\vdash Q \neq \top \to (P \vee \floor*{Q} = \top)
\end{equation}
by Corollary~\ref{cor:classificationreasoning}, and the fact that $\vdash P \vee \floor*{Q} = \top$ and $\vdash \top$ imply $\vdash P \vee \floor*{Q}$. 

The proof of~\eqref{eq:predicatevee-01} is straightforward as follows.
\begin{align*}
&\vdash Q = \top \to (P \vee \floor*{Q} = \top) \\
\text{if} \quad &\vdash Q = \top \to (P \vee \floor*{\top} = \top) \\
\text{if} \quad &\vdash Q = \top \to (\top = \top) \\
\text{if} \quad &\vdash \top.
\end{align*}

The proof of~\eqref{eq:predicatevee-02} needs more effort:
\begin{align*}
&\vdash Q \neq \top \to (P \vee \floor*{Q} = \top) \\
\text{iff} \quad &\vdash (Q = \top) \vee (P \vee \floor*{Q} = \top) \\
\text{iff} \quad &\vdash (\floor*{Q} = \top) \vee (P \vee \floor*{Q} = \top) \\
\text{iff} \quad &\vdash \floor*{Q} \neq \top \to (P \vee \floor*{Q} = \top) \\
\text{iff} \quad &\vdash \floor*{Q} = \bot \to (P \vee \floor*{Q} = \top) \\
\text{if} \quad &\vdash \floor*{Q} = \bot \to (P \vee \bot = \top) \\
\text{iff} \quad &\vdash \floor*{Q} = \bot \to (P = \top) \\
\text{if} \quad &\vdash Q = \top \vee P = \top.
\end{align*}
Notice that $P$ is a predicate pattern, so it suffices to show
\begin{equation*}
\vdash P = \top \to (Q = \top \vee P = \top),
\end{equation*}
whose validity is obvious, and
\begin{equation*}
\vdash P = \bot \to (Q = \top \vee P = \top),
\end{equation*}
which is proved by showing
\begin{equation} 
\vdash P = \bot \to Q = \top.
\end{equation}
Because $\vdash P \vee Q$, it suffices to show
\begin{align*}
&\vdash P = \bot \to (P \vee Q) \to (Q = \top) \\
\text{if} \quad &\vdash P = \bot \to (\bot \vee Q) \to (Q = \top) \\
\text{iff} \quad &\vdash P = \bot \to Q \to (Q = \top) \\
\text{if} \quad &\vdash Q \to (Q = \top) \\
\text{iff} \quad &\vdash (Q \neq \top) \to \neg Q \\
\text{iff} \quad &\vdash (\floor*{Q} = \bot) \to \neg Q.
\end{align*}
Notice we have $\vdash Q \to \floor*{Q}$, which means $\vdash \neg \floor*{Q} \to \neg Q$, so it suffices to show
\begin{align*}
&\vdash (\floor*{Q} = \bot) \to \neg \floor*{Q} \\
\text{iff} \quad &\vdash (\floor*{Q} = \bot) \to \neg \bot \\
\text{iff} \quad &\vdash (\floor*{Q} = \bot) \to \top \\
\text{iff} \quad &\vdash \top.
\end{align*}
And this concludes the proof.
\end{proof}


\begin{prop}[Deduction Theorem]
If $\Gamma \cup \{P\} \vdash Q$ and the derivation does not use \mbox{$\forall x$-Generalization} where $x$ is free in $P$, then $\Gamma \vdash \floor*{P} \to Q$. 
\end{prop}
\begin{proof}
The proof is by induction on $n$, the length of the derivation of $Q$ from $\Gamma \cup \{P\}$.

Base step: $n = 1$, and $Q$ is an axiom, or $P$, or a member of $\Gamma$. If $Q$ is an axiom or a member of $\Gamma$, then $\Gamma \vdash Q$ and as a result, $\Gamma \vdash \floor*{P} \to Q$. If $Q$ is $P$, then $\Gamma \vdash \floor*{P} \to Q$ by Proposition~\ref{prop:p->ceilp}.

Induction step: Let $n > 1$. Suppose that if $P'$ can be deduced from $\Gamma \cup \{P\}$ without using $\forall x$-Generalization where $x$ is free in $P$, in a derivation containing fewer than $n$ steps, then $\Gamma \vdash \floor*{P} \to P'$.

Case 1: $Q$ is an axiom, or $P$, or a member of $\Gamma$. Precisely as in the Base step, we show that $\vdash \floor*{P} \to Q$.

Case 2: $Q$ follows from two previous patterns in the derivation by an application of Modus Ponens. These two patterns must have the forms $Q_1$ and $Q_1 \to Q$, and each one can certainly be deduced from $\Gamma \cup \{P\}$ by a derivation with fewer than $n$ steps, by just omitting the subsequent members from the original derivation from $\Gamma \cup \{P\} \vdash Q$. So we have $\Gamma \cup \{P\} \vdash Q_1$ and $\Gamma \cup \{P\} \vdash Q_1 \to Q$, and, applying the hypothesis of induction, $\Gamma \vdash \floor*{P} \to Q_1$ and $\Gamma \vdash \floor*{P} \to (Q_1 \to Q)$. It follows immediately that $\Gamma \vdash \floor*{P} \to Q$.

Case 3: $Q$ follows from a previous pattern in the derivation by an application of $\forall x_i$-Generalization where $x_i$ does not occur free in $P$. So $Q$ is $\forall x_i . Q_1$, say, and $Q_1$ appears previously in the derivation. Thus $\Gamma \cup \{P\} \vdash Q_1$, and the derivation has fewer than $n$ steps, so $\Gamma \vdash \floor*{P} \to Q_1$, since there is no application of Universal Generalization involving a free variable of $P$. Also $x_i$ cannot occur free in $P$, as it is involved in an application of Universal Generalization in the deduction of $Q$ from $\Gamma \cup \{P\}$. So we have a derivation of $\Gamma \vdash \floor*{P} \to Q$ as follows.
\begin{align*}
\Gamma &\vdash \floor*{P} \to Q \\
{\text{iff} \quad} \Gamma &\vdash \floor*{P} \to \forall x_i . Q_1 \\
{\text{if} \quad} \Gamma &\vdash \forall x_i . (\floor*{P} \to Q_1) \\
{\text{if} \quad} \Gamma &\vdash \floor*{P} \to Q_1. \\
\end{align*}
So $\Gamma \vdash \floor*{P} \to Q$ as required.

Case 4: Q follows from a previous pattern in the derivation by an application of Membership Introduction. So $Q$ is $\forall x_i . (x_i \in Q_1)$ with $x_i$ is free in $Q_1$, say, and $Q_1$ appears previously in the derivation. Thus $\Gamma \cup \{P\} \vdash Q_1$, and the derivation has fewer than $n$ steps, so $\Gamma \vdash \floor*{P} \to Q_1$, since there is no application of Universal Generalization involving a free variable of $P$. So we have a derivation of $\Gamma \vdash \floor*{P} \to Q$ as follows.
\begin{align*}
\Gamma &\vdash \floor*{P} \to Q \\
{\text{iff} \quad} \Gamma &\vdash \floor*{P} \to \forall x_i . (x_i \in Q_1) \\
{\text{iff} \quad} \Gamma &\vdash \floor*{P} \to \floor*{Q_1}, \\
\end{align*}
which follows by the hypothesis of induction $\Gamma \vdash \floor*{P} \to Q_1$ and the fact that $\Gamma \vdash Q_1 \to \floor*{Q_1}$ (by the Remark in Proposition~\ref{prop:p->ceilp}).

Case 5: Q follows from a previous pattern in the derivation by an application of Membership Elimination. The previous pattern must have the form $\forall x_i . (x_i \in Q)$, and can be deduced from $\Gamma \cup \{P\}$ by a derivation with fewer than $n$ steps, by just omitting the subsequent members from the original derivation from $\Gamma \cup \{P\} \vdash Q$. So we have $\Gamma \cup \{P\} \vdash \forall x_i . (x_i \in Q)$, and, applying the hypothesis of induction, $\Gamma \vdash \floor*{P} \to \forall x_i . (x_i \in Q)$. So we have a derivation of $\Gamma \vdash \floor*{P} \to Q$ as follows.
\begin{align*}
\Gamma &\vdash \floor*{P} \to Q \\
{\text{iff} \quad} \Gamma &\vdash \neg \floor*{P} \vee Q \\
{\text{iff} \quad} \Gamma &\vdash \neg \floor*{P} \vee \floor*{Q} \tag{Proposition~\ref{prop:predicatevee}} \\
{\text{iff} \quad} \Gamma &\vdash \neg \floor*{P} \vee \forall x_i . (x_i \in Q) \\
{\text{iff} \quad} \Gamma &\vdash \floor*{P} \to \forall x_i . (x_i \in Q),
\end{align*}
which is the hypothesis of induction. And this concludes our inductive proof.
\end{proof}

\begin{cor}[Closed-form Deduction Theorem]
If $P$ is closed, $\Gamma \cup \{P\} \vdash Q$ implies $\Gamma \vdash \floor*{P} \to Q$.
\end{cor}

\begin{thm} [Frame Rule]
	Let $\sigma \in \Sigma$ be a symbol in the signature. From $P_1 \to P_2$, deduce $\sigma(P_1) \to \sigma(P_2)$. In its most general form, $P_1 \to P_2$ deduces $\sigma(Q_1,\dots,P_1,\dots,Q_n) \to \sigma(Q_1,\dots, P_2, \dots, Q_n)$.
\end{thm}
\begin{proof}
	we write $\sigma(Q_1,\dots,P_i,\dots,Q_n)$ as $\sigma(P_i,\vec{Q})$ for short, for any $i \in \{1,2\}$.
	\begin{align*}
	&\vdash \sigma(P_1, \vec{Q}) \to \sigma(P_2,\vec{Q}) \\
	\text{iff} \quad &\vdash y \in (\sigma(P_1, \vec{Q}) \to \sigma(P_2,\vec{Q})) \\
	\text{iff} \quad &\vdash (y \in \sigma(P_1, \vec{Q})) \to (y \in \sigma(P_2, \vec{Q})) \\
	\text{iff} \quad &\vdash \exists z_1 . \exists \vec{z} . (z_1 \in P_1 \wedge \vec{z} \in \vec{Q} \wedge y \in \sigma(z_1, \vec{z})) \\
	&\to \exists z_2 . \exists \vec{z} . (z_2 \in P_2 \wedge \vec{z} \in \vec{Q} \wedge y \in \sigma(z_2, \vec{z})) \\
	\text{iff} \quad &\vdash \exists z_1 . \exists \vec{z} . (z_1 \in P_1 \wedge \vec{z} \in \vec{Q} \wedge y \in \sigma(z_1, \vec{z}) \\
	 &\quad \quad \quad \to z_1 \in P_2 \wedge \vec{z} \in \vec{Q} \wedge y \in \sigma(z_1, \vec{z})) \\
	\text{iff} \quad &\vdash \exists z_1. \exists \vec{z} . (z_1 \in P_1 \to z_1 \in P_2) \\
	\text{if} \quad &\vdash \exists z_1 .  (z_1 \in P_1 \to z_1 \in P_2) \\
	\text{if} \quad &\vdash P_1 \to P_2.
	\end{align*}
\end{proof}
\begin{cor} [Frame Rule as Implication]
	$\vdash \floor*{P \to Q} \to (\sigma(P) \to \sigma(Q))$
\end{cor}

\section{Inference rules}

\paragraph{Axioms}
$$
\infer{\Gamma \vdash A}{\cdot}
$$
where $A$ is an axiom. 

\paragraph{Inclusion}
$$
\infer{\Gamma \vdash P}{\cdot}
$$
where $P \in \Gamma$.

\paragraph{Modus Ponens}
$$
\infer{\Gamma \vdash P}
      {\Gamma \vdash Q \to P
      &\Gamma \vdash Q}
$$

\paragraph{Closed-Form Deduction Theorem}
$$
\infer{\Gamma \vdash P \to Q}
{\Gamma \cup \{P\} \vdash Q}
$$
where $P$ is closed.

\paragraph{Universal Generalization}
$$
\infer[(\forall x)]{\Gamma \vdash \forall x . P}
      {\Gamma \vdash P}
$$

\paragraph{Conjunction Splitting}
$$
\infer{\Gamma \vdash P \wedge Q}
      {\Gamma \vdash P
      &\Gamma \vdash Q}
$$

\end{document}
