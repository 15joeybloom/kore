\documentclass{article}
\usepackage{amsmath}
\usepackage{amsthm}
\usepackage{amssymb}
\usepackage[utf8]{inputenc}
\usepackage{proof}
\usepackage{txfonts}
\usepackage{mathtools}

\newtheorem{thm}{Theorem}
\newtheorem{prop}[thm]{Proposition}
\newtheorem{cor}[thm]{Corollary}
\newtheorem{defn}[thm]{Definition}

\DeclarePairedDelimiter\ceil{\lceil}{\rceil}
\DeclarePairedDelimiter\floor{\lfloor}{\rfloor}

\def\imp{\to}
\def\land{\mathbin\&}

\title{Towards an Efficient and Economic Deductive System of Matching Logic}
\author{FSL group}



\begin{document}
\maketitle
We aim for a Hilbert style deductive system which has a relatively large number of axioms but only a few inference rules. 

\section{Grammar and extended grammar}

The formal language $\mathcal{L}$ we use to write matching logic patterns is defined as follows.
\begin{align*}
P \Coloneqq\  & x \\
       \mid\  & P_1 \to P_2 \\
       \mid\  & \neg P \\
       \mid\  & \forall x . P \\
       \mid\  & \sigma(P_1,\dots,P_n) \\
 {\tt (*** }\ & \ {\tt  extended  ***)} \\
       \mid\  & P_1 \vee P_2 \\
       \mid\  & P_1 \wedge P_2 \\
       \mid\  & P_1 \leftrightarrow P_2 \\
       \mid\  & \exists x . P \\
       \mid\  & \ceil*{P} \\
	   \mid\  & \floor*{P} \\
       \mid\  & P_1 = P_2 \\
       \mid\  & P_1 \neq P_2 \\
       \mid\  & \top \\
       \mid\  & \bot \\
       \mid\  & P_1 \subseteq P_2 \\
       \mid\  & x \in P
\end{align*}
with the extended grammar defined as
\begin{align*}
&P_1 \vee P_2 \coloneqq \neg P_2 \to P_1 \\
&P_1 \wedge P_2 \coloneqq \neg (\neg P_1 \vee \neg P_2) \\
&P_1 \leftrightarrow P_2 \coloneqq (P_1 \to P_2) \wedge (P_2 \to P_1) \\
&\exists x . P \coloneqq \neg \forall x . \neg P \\
&\floor*{P} \coloneqq {\neg} \ceil*{\neg P} \\
&P_1 = P_2 \coloneqq \floor*{P_1 \leftrightarrow P_2} \\
&P_1 \neq P_2 \coloneqq \neg (P_1 = P_2) \\
&\bot \coloneqq x_1 \wedge \neg x_1 \\
&\top \coloneqq {\neg} \bot \\
&P_1 \subseteq P_2 \coloneqq \floor*{P_1 \to P_2} \\
&x \in P \coloneqq \ceil*{x \wedge P}
\end{align*}

We will extend the grammar to a many-sorted one in the future.

\section{Hilbert proof system}
Axioms in $\mathcal{L}$ are given by the following nine axiom schemata where $P, Q, R$ are arbitrary patterns and $x, y$ are variables.
\begin{itemize}
\item (K1) $P \to (Q \to P)$
\item (K2) $(P \to (Q \to R)) \to ((P \to Q) \to (P \to R))$
\item (K3) $(\neg P \to \neg Q) \to (Q \to P)$
\item (K4) $\forall x . (P \to Q) \to (P \to \forall x . Q)$ if $x$ does not occur free in $P$
\item (K5) $\exists y . x = y$
\item (K6) $\exists y . Q = y \to (\forall x . P(x) \to P[Q/x])$ if $Q$ is free for $x$ in $P$
\item (K7) $P_1 = P_2 \to (Q[P_1/x] \to Q[P_2/x])$
\item (M1) $x \in y = (x = y)$
\item (M2) $x \in \neg P = \neg (x \in P)$
\item (M3) $x \in P \wedge Q = (x \in P) \wedge (x \in Q)$
\item (M4) $x \in \exists y . P = \exists y . x \in P$ where $x$ is distinct from $y$
\item (M5) $x \in \sigma(\dots,P_i,\dots) = \exists y . y \in P_i \wedge x \in \sigma(\dots,y,\dots)$
\end{itemize}

Inference rules include
\begin{itemize}
\item (Modus Ponens) From $P$ and $P \to Q$, deduce $Q$.
\item (Universal Generalization) From $P$, deduce $\forall x . P$. 
\item (Membership Introduction) From $P$, deduce $\forall x . x \in P$, where $x$ occurs free in $P$.
\item (Membership Elimination) From $\forall x . x \in P$, deduce $P$, where $x$ occurs free in $P$.
\end{itemize}

\begin{thm}[Soundness of $K_\mathcal{L}$]
Theorems of $K_\mathcal{L}$ are valid.
\end{thm}
\begin{proof}
	Trivial.
\end{proof}

We provide some metatheorems of $K_\mathcal{L}$.

\begin{prop}[Tautology] \label{prop:taut}
	For any propositional tautology $\mathcal{A}(p_1,\dots,p_n)$ where $p_1,\dots,p_n$ are propositional variables, $$ \vdash \mathcal{A}(P_1,\dots,P_n).$$
\end{prop}
\begin{proof}
	Omit proof here.
\end{proof}
\paragraph{Remark} Proposition~\ref{prop:taut} makes any metatheorem of propositional logic a metatheorem of $K_\mathcal{L}$.
\begin{prop} [Variable Substitution]
	$\vdash \forall x . P \to P[y / x]$.
\end{prop}
\begin{prop}[Functional Substitution]
	$\vdash \exists y . (Q = y) \to (P[Q / x] \to \exists x . P(x))$.
\end{prop}
\begin{prop}[$\vee$-Introduction]
	$\vdash P$ implies $\vdash P \vee Q$.
\end{prop}
\begin{proof}
Use Proposition~\ref{prop:taut} and Modus Ponens.
Note that in general, $\vdash P \vee Q$ does not imply $\vdash P$ or $\vdash Q$.
\end{proof}

\begin{prop}[$\wedge$-Introduction and Elimination]
\label{prop:wedge}
$\vdash P$ and $\vdash Q$ iff $\vdash P \wedge Q$.
\end{prop}
\begin{proof}
Use Proposition~\ref{prop:taut} and Modus Ponens.
\end{proof}

\begin{prop}[Equality Introduction]
	$\vdash P = P$.
\end{prop}
\begin{proof}
	Use Membership Introduction and Proposition~\ref{prop:taut}.
\end{proof}

\begin{prop}[Equality Replacement]
\label{prop:ereplacement}
$\vdash P_1 = P_2$ and $\vdash Q[P_1 / x]$ implies $\vdash Q[P_2 / x]$.
\end{prop}
\begin{proof}
	Use Axiom~(K7) and Modus Ponens.
\end{proof}

\begin{prop}[Equality Establishment]
\label{prop:eestablish}
$\vdash P \leftrightarrow Q$ implies $\vdash P = Q$.
\end{prop}
\begin{proof}
	Use Membership Axoims and $\vee$-Introduction.
\end{proof}
\begin{cor} \label{cor:top}
$\vdash P$ implies $\vdash P = \top$ .
\end{cor}

\begin{prop} \label{prop:var1}
$\vdash x \in \ceil*{y}$.
\end{prop}
\begin{proof}
\begin{align*}
&\vdash x \in \ceil*{y} \\
{\text{if}}& \vdash \forall x . (x \in \ceil*{y}) \tag{$K5$, $K6$, and Modus Ponens} \\
{\text{iff}}& \vdash \ceil*{y}.
\end{align*}
\end{proof}

\begin{prop}
$\vdash P \to \ceil*{P}$.
\end{prop}
\begin{proof}
\begin{align*}
&\vdash P \to \ceil*{P} \\
{\text{iff}}& \vdash \forall x . (x \in P \to \ceil*{P}) \\
{\text{if}}&  \vdash x \in P \to \ceil*{P} \\
{\text{iff}}& \vdash x \in P \to x \in \ceil*{P} \\
{\text{iff}}& \vdash x \in P \to \exists y . (y \in P \wedge x \in \ceil*{y}) \\
{\text{iff}}& \vdash x \in P \to \neg \forall y . (y \not \in P \vee x \not \in \ceil*{y}) \\
{\text{iff}}& \vdash \forall y . (y \not \in P \vee x \not \in \ceil*{y}) \to x \not \in P \\
{\text{if}}&  \vdash x \not \in P \vee x \not \in \ceil*{x} \to x \not \in P \\
{\text{iff}}& \vdash x \in P \to x \in P \wedge x \in \ceil*{x} \\
{\text{iff}}& \vdash x \in P \to x \in \ceil*{x} \\
{\text{if}}&  \vdash x \in \ceil*{x} \\
\end{align*}
\paragraph{Remark} Similarly we can show $\vdash \floor*{P} \to P$.
\end{proof}

\begin{prop}
$\vdash \forall x . (x \in P) = \floor*{P}$, where $x$ occurs free in $P$.
\end{prop}
\begin{proof}
By Proposition~\ref{prop:eestablish} and~\ref{prop:wedge}, it suffices to show \begin{equation} \label{eq:prop11-1}
\vdash \forall x . (x \in P) \to \floor*{P}
\end{equation}
and
\begin{equation} \label{eq:prop11-2}
\vdash \floor*{P} \to \forall x . (x \in P).
\end{equation}

To show~\eqref{eq:prop11-1},

\begin{align*}
&\vdash \forall x . (x \in P) \to \floor*{P} \\
{\text{iff}}& \vdash \forall x . \ceil*{x \wedge P} \to \neg \ceil*{\neg P} \\
{\text{iff}}& \vdash \ceil*{\neg P} \to \exists x . \neg \ceil*{x \wedge P} \\
{\text{iff}}& \vdash \forall y . (y \in ( \ceil*{\neg P} \to \exists x . \neg \ceil*{x \wedge P})) \\
{\text{if}}& \vdash y \in ( \ceil*{\neg P} \to \exists x . \neg \ceil*{x \wedge P} \\
{\text{iff}}& \vdash \exists z_1 . (z_1 \not \in P \wedge y \in \ceil*{z_1}) \to \\ & \quad \quad \exists x . \neg (\exists z_2 . (z_2 = x \wedge z_2 \in P \wedge y \in \ceil*{z_2})) \\
{\text{iff}}& \vdash \exists z_1 . (z_1 \not \in P \wedge \top) \to \tag{Proposition~\ref{prop:var1}, \ref{prop:ereplacement}, and Corollary~\ref{cor:top}}\\
& \quad \quad \exists x . \neg (\exists z_2 . (z_2 = x \wedge z_2 \in P \wedge \top)) \\
{\text{iff}}& \vdash \exists z_1 . (z_1 \not \in P) \to \exists x . \neg (\exists z_2 . (z_2 = x \wedge z_2 \in P)) \\
{\text{iff}}& \vdash \forall x . (\exists z_2 . (z_2 = x \wedge z_2 \in P)) \to \forall z_1 . (z_1 \in P) \\
{\text{if}}& \vdash \forall z_1 . (\forall x . (\exists z_2 . (z_2 = x \wedge z_2 \in P)) \to (z_1 \in P)) \\
{\text{if}}& \vdash \forall x . (\exists z_2 . (z_2 = x \wedge z_2 \in P)) \to (z_1 \in P).
\end{align*}
Since $\vdash \forall x . (\exists z_2 . (z_2 = x \wedge z_2 \in P)) \to \exists z_2 . (z_2 = z_1 \wedge z_2 \in P) $, it suffices to show
\begin{align*}
&\vdash \exists z_2 . (z_2 = z_1 \wedge z_2 \in P) \to (z_1 \in P) \\
{\text{iff}}& \vdash z_1 \not \in P \to \forall z_2 . (z_2 \neq z_1 \vee z_2 \not \in P) \\
{\text{if}}& \vdash \forall z_2 . (z_1 \not \in P \to z_2 \neq z_1 \vee z_2 \not \in P) \\
{\text{if}}& \vdash z_1 \not \in P \to z_2 \neq z_1 \vee z_2 \not \in P \\
{\text{if}}& \vdash z_2 = z_1 \wedge z_2 \in P \to z_1 \in P.
\end{align*}
And we proved~\eqref{eq:prop11-1}.

Similarly, to show~\eqref{eq:prop11-2}, 
\begin{align*}
&\vdash \floor*{P} \to \forall x . (x \in P)  \\
{\text{iff}}& \vdash \exists x . \neg \ceil*{x \wedge P} \to \ceil*{\neg P} \\
{\text{iff}}& \vdash \forall y . (y \in \exists x . \neg \ceil*{x \wedge P} \to \ceil*{\neg P}) \\
{\text{if}}& \vdash y \in \exists x . \neg \ceil*{x \wedge P} \to \ceil*{\neg P} \\
{\text{iff}}& \vdash \exists x . \neg \exists z_2 . (z_2 = x \wedge z_2 \in P) \to \exists z_1 . (z_1 \not \in P) \\
{\text{iff}}& \vdash \forall z_1 . (z_1 \in P) \to \exists z_2 . (z_2 = x \wedge z_2 \in P)\\
{\text{iff}}& \vdash x \in P \to \exists z_2 . (z_2 = x \wedge z_2 \in P).
\end{align*}
We proved~\eqref{eq:prop11-2}.

\paragraph{Remark} If $x$ occurs free in $P$, the result does not hold. For example, let $P$ be $upto(x)$ where $upto(\cdot)$ is interpreted to $upto(n) = \{0, 1, \dots, n\}$ on $\mathbb{N}$. 
\end{proof}

\begin{prop}[Deduction Theorem]
If $\Gamma \cup \{P\} \vdash Q$ and the proof does not use \mbox{$\forall x$-Generalization} where $x$ is free in $P$, then $\Gamma \vdash P \to Q$. In particular, when $P$ is closed, $\Gamma \cup \{P\} \vdash Q$ implies $\Gamma \vdash P \to Q$.
\end{prop}



\begin{prop}[More Theorems in $\mathcal{L}$]
\begin{align*}
&\vdash \exists x . x \\
&\vdash \ceil*{x} \\
&\vdash \exists y . x = y \\
&\vdash P_1 = P_2 \to Q[P_1 / x] = Q[P_2 / x]
\end{align*}
\end{prop}
\begin{proof}
(1)
$$
\infer[^{\tt (DEF)}]{\vdash \exists x . x}
{\infer[^{\tt (MP)}]{\vdash \neg \forall x . \neg x}
	{\infer[^{\tt (TAUT)}]{\vdash \neg (x \to \neg x)}{\cdot}
		&{\infer[^{\tt (MP\&K3)}] 
	{\vdash \neg (x \to \neg x) \to \neg \forall x . \neg x}
	{\infer[^{\tt (DEDUCT)}]
	{\vdash (\forall x . \neg x) \to (x \to \neg x)}
	{\infer[^{\tt (MP)}]{\forall x . \neg x \vdash x \to \neg x}
		{{\infer{\forall x . \neg x \vdash \neg x}
{\forall x . \neg x \vdash \neg x
&\forall x . \neg x \vdash \neg x}}
	    &{\infer[^{\tt (K1)}]{\forall x . \neg x \vdash \neg x \to (x \to \neg x)}{\cdot}}
}}}}}}
$$
(3)
$$
\infer[^{\tt (DEF)}]{\vdash \exists y . x = y}
      {\infer[^{\tt (MP)}]{\vdash \neg \forall y . \neg (x = y)}
      	     {\infer[^{\tt (MP\&TAUT)}]{\vdash \neg \neg (x = x)}
      	     	    {\infer[^{\tt (K7)}]{x = x}{\cdot}}
             &\infer[^{\tt (MP\&K3)}]{\vdash \neg \neg (x = x) \to  \neg \forall y . \neg (x = y)}
                    {\infer[^{\tt {(K6)}}]{\vdash \forall y . \neg (x = y) \to  \neg (x = x)}{\cdot}}}}
$$

\end{proof}

Before we prove the adequacy theorem of $K_\mathcal{L}$, we prove some lemmas.

\begin{prop}
If a pattern is valid, then its closure is valid.
\end{prop}

\begin{prop}
If a pattern's closure is a theorem of $K_\mathcal{L}$, then itself is a theorem of $K_\mathcal{L}$, too.
\end{prop}

\begin{prop}
If $P$ is not a theorem of $K_\mathcal{L}$, then $K_\mathcal{L}$ extended with adding $\neg P$ as an axiom is consistent.
\end{prop}

\begin{prop}
If $S$ is a consistent extended system of $K_\mathcal{L}$, then for any theorem $P$ of $S$, there exists a model $M$ and an $M$-evaluation $\rho$ such that $M,\rho \vDash P$.
\end{prop}

\section{Inference rules}

\paragraph{Axioms}
$$
\infer{\Gamma \vdash A}{\cdot}
$$
where $A$ is an axiom. 

\paragraph{Inclusion}
$$
\infer{\Gamma \vdash P}{\cdot}
$$
where $P \in \Gamma$.

\paragraph{Modus Ponens}
$$
\infer{\Gamma \vdash P}
      {\Gamma \vdash Q \to P
      &\Gamma \vdash Q}
$$

\paragraph{Closed-Form Deduction Theorem}
$$
\infer{\Gamma \vdash P \to Q}
{\Gamma \cup \{P\} \vdash Q}
$$
where $P$ is closed.

\paragraph{Universal Generalization}
$$
\infer[(\forall x)]{\Gamma \vdash \forall x . P}
      {\Gamma \vdash P}
$$

\paragraph{Conjunction Splitting}
$$
\infer{\Gamma \vdash P \wedge Q}
      {\Gamma \vdash P
      &\Gamma \vdash Q}
$$

\end{document}
